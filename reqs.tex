
\documentclass[12pt, letterpaper]{article}
\usepackage{graphicx}

\pdfpagewidth=8.5in
\pdfpageheight=11in

\title{Requirements}
\author{our names}
\date{September 2022}

\pagenumbering{arabic}

\begin{document}
\maketitle
\pagestyle{plain}

\title{Requirements}

\section{Introduction}

\section{Solar System Thing}
%
\textbf{Pump Monitor}
%
\textit{CIU 903 SQFlex control unit for remote monitoring}
%
"When combined with a CIM 280 GRM module, it enables remote monitoring through Grundfos Go".
%
"CIU 903 can be connected to a start/stop switch, a level switch or a pulsating water meter"
%
Can do system monitoring and alarm indication.
%
Monitors water reservoir full, pump running, input power.
%
Alarms for dry running, "service needed".

\textit{Dry-running protection}
%
A water-level electrode is used to measure water levels.
%
When water level drops below the electrode, the pump stops.

\textbf{Batteries!!!}
%
\textit{SQFlex Solar with backup batteries:} Backup batteries can be connected to the PV system
to charge.
%
This is typically used to power the pump in the case of solar panels not getting enough sun???
%
"Power from the solar panels is fed into a 48 VDC charge controller, which regulates the current fed to the batteries."
%
"From the charge controller, power passes into the battery bank, which consists of the number of
appropriately sized batteries wired in series to achieve 48 VDC (rated) output".


\section{Related Work}
\subsection{Water Monitoring}
\textbf{Paper Cenek sent}
%

\textit{Purpose:} monitor water flow inside homes in rural Alaska; monitor water flow to different features in house.
%

\textit{Design goals:} dependability, manufacturability, cost, minimal time to install, minimal maintenance, portability.
%

\textit{Methods:} used alkaline batteries. Flow meters to measure when water was flowing, vibration sensors attached to each
sensor to measure which feature is drawing current.
%

\textit{Vibration sensor node setup:} Moteino on vibration sensors; included power supply circuit board, 6 AA batteries,
custom amplifier, connected to piezoelectric element; total cost of \$65 per sensor setup. Data is transmitted from
the sensors to the base station using Moteino.
%
Power: two sets of 3 AA batteries power the sensor nodes; Power Shield with boost/buck voltage convert is used
to interface batteries with Moteino (this board provides a voltage reading to determine when to replace batteries);
use this to determine maintenance for when to replace batteries.
%
Piezoelectric element used to convert vibrations to AC voltage.

\textit{Flow meter setup:} Raspberry Pi supplies power to flow meters.
%

\textit{Base station setup:} Raspberry Pi is used at central base station.
%
Central base station receives all information transmitted from sensors and flow meter.
%
To receive signal from Moteino sensors, it has a MoteinoUSB.
%
Has Ethernet connected modem and custom GPIO interface.
%
All enclosed in a PVC junction box.
%
Total cost was \$205.
%
Power: AC power supply.
%
GPIO was used to connect to a Real Time Clock (RTC) for timestamping, connect to flowmeter.
%
Uses a Moteino as a hardware-based watchdog to perform a hard reboot of Raspberry Pi if necessary.
%
A significant amount of data processing/data aggregation is done at the ground station.
%
They send the data files daily over email.
%
They give two SD cards for the raspberry pis; if the case of a corrupted SD card in a pi, the backup
card can be used as a replacement; ***the key thing here is during the install process, we have to
calibrate the first sd card on the pi with all the sensors then make a duplicate of this sd card
on the backup.

\textit{Data aggregation:} There is a lot of information on data aggregation in section 1.2.3.
%
I didn't read most of it!
%
There's a helpful section on how they handle if the cell service is down and data cannot be transmitted.
%
Someone should look at their data file transmission process figure, it seems helpful.

\textit{Testing} Tested sensors before deployment to determine "amount of capacity used"...
%
I think that's all they detailed in there.


\textit{Considerations based on this project:}
%
\begin{itemize}
%
\item Need method to hard reboot rasberry pi in case of failure.
\item Should reboot rasberry pi every 24 hours to prevent device system crash.
\item Need backup SD card for each pi that is clone of primary card.
%
\end{itemize}


\section{Design Goals}
%
\subsection{Primary Design Goals}
%
\textit{*minimum viable product}
%
\begin{itemize}
%
\item Easy maintenance
%
\item Minimal install time
%
\item Reliability
%
\end{itemize}
%

\subsection{Secondary Design Goals}
%
\begin{itemize}
%
\item Energy efficiency
%
\end{itemize}


\section{Required Materials}
\begin{enumerate}
\item raspberry pi. \textit{considerations: raspberry pi's cannot perform hard reboots themselves,
so we will need to install something for that.}
\item Real time clock
\end{enumerate}

\subsection{Materials we cannot use}
\begin{enumerate}
\item \textit{Lithium ion batteries} FAA restricts traveling with these. Cite: https://www.faa.gov/hazmat/packsafe/
\end{enumerate}

\end{document}
